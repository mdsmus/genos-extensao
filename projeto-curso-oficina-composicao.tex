\documentclass[12pt]{article}
\usepackage{graphicx}
\usepackage[utf8,utf8x]{inputenc}
\usepackage[T1]{fontenc}
\usepackage[brazil]{babel}
\usepackage{url}
\usepackage[a4paper,top=2.5cm,left=3cm,right=3cm,bottom=2.5cm]{geometry}
\usepackage[perpage,symbol*,para]{footmisc}
\usepackage{setspace}

\newcommand{\sigla}[1]{\textsc{#1}}
\newcommand{\eng}[1]{\textit{#1}}
\newcommand{\prog}[1]{\textsf{#1}}

\setlength\parindent{0cm}

\begin{document}

\title{Oficina de Composição musical a partir de recursos eletrônicos}
\author{Marcos di Silva, Cristiano Figueiró e Guilherme Bertissolo}
\maketitle

\thispagestyle{empty}

\section{Introdução}

%% desenvolver
Uma disciplina que aborde a relação entre composição musical e
recursos eletrônicos é importante.

Esta disciplina exige investimento tanto em recursos materiais quanto
humanos.

\section{Ementa}

Estudo de princípios da acústica musical, de áudio digital, de
sequenciamento e notação MIDI, de práticas de gravação, de mixagem,
espacialização sonora e equalização, de estação de trabalho para áudio
digital, de software livre para áudio, de microfones, interfaces e
suportes. Panorama de repertório de música criada com recursos
eletrônicos e breve história de recursos eletrônicos em música.

\section{Conteúdo programático}

\begin{enumerate}
\item Introdução à acústica musical
\item MIDI: Sequenciamento e notação
\item Introdução ao áudio digital
\item Microfones, interfaces e suportes
\item Introdução a práticas de gravação
\item Mixagem, espacialização sonora e equalização
\item Introdução a estação de trabalho para áudio digital
\item Software livre para áudio
\item Breve história de recursos eletrônicos em música
\item Panorama dos repertórios de música criada com recursos eletrônicos
\end{enumerate}

\section{Carga horária}

34 horas
 
\section{Módulo}

Mínimo de 20 e máximo de 40 alunos. O cálculo do módulo é de 1 a 2
alunos por computador.

\section{Recursos necessários}
\label{sec:material}

\subsection{Recursos materiais}

\begin{enumerate}
\item Sala de aula com 100 m$^2$ acusticamente preparada e com ar
  condicionado
\item 20 computadores com placa de som profissional (10x10) e monitor
  de vídeo LCD de 20 polegadas
\item 40 Headphones profissionais
\item 20 controladores MIDI
\item Sistema de monitores de áudio 5.1
\item Mesa de som automatizada de 24 canais
\item Patchbay
\item Microfones condensador
\item Microfones
\item Laptop com placa de som profissional (10x10)
\end{enumerate}

\subsection{Recursos humanos}

\begin{enumerate}
\item 1 professor com formação em música, conhecimento aprofundado em
  software livre, composição e em recursos eletrônicos em música
\item 1 técnico especialista em áudio e informática (bolsista)
\item 2 monitores
\end{enumerate}

\end{document}
