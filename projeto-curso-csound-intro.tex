\documentclass[12pt,brazil]{article}
\usepackage[utf8x]{inputenc}
\usepackage{babel}
\usepackage[top=25mm,bottom=25mm,left=25mm,right=25mm]{geometry}

\newcommand{\sigla}[1]{\textsc{#1}}
\newcommand{\eng}[1]{\textit{#1}}
\newcommand{\prog}[1]{\textsf{#1}}

\begin{document}
\selectlanguage{brazil}

\title{Curso introdutório de Csound}
\author{Pedro Kröger}
\maketitle

\begin{abstract}
O Csound pode ser considerado atualmente como um dos melhores e mais
poderosos programas para síntese sonora, ele é excelente para o design
de instrumentos, é multi-plataforma, tem o código fonte aberto, e seu
formato não é proprietário.

O objetivo é oferecer uma nova edição do curso ``Introdução ao
Csound'' ministrado por Pedro Kröger em 2001 e 2003. Além disso,
fornecer os conceitos e técnicas básicas de síntese sonora, entender a
teoria básica e limitações da representação digital do som, aprender
os conceitos fundamentais de computação e programação, aprender a
linguagem de síntese Csound e como usá-la. O curso terá 5 aulas de 2
horas.

O curso se dirige aos alunos de composição musical, mas alunos de
outras áreas (inclusive não musicais, como ciência da computação) são
bem vindos.  Os alunos devem estar cursando ou ter concluído o curso
superior.  O conhecimento básico de como utilizar um sistema
operacional (Windows, MacOs, Linux) é requerido.

Pedro Kröger tem trabalhado com o Csound há cerca de 12 anos
elaborando não apenas composições, mas também utilizando seu potencial
para palestras didáticas, além de desenvolver tutoriais e ferramentas.
Ele contribuiu com o desenvolvimento da versão ``não-oficial'' do
Csound para Linux, mantida por Nicola Bernardini. Sua tese de
doutorado está centrada no desenvolvimento de tecnologias e
ferramentas para expandir o Csound.
\end{abstract}

\thispagestyle{empty}

\section{Introdução}
O Csound tem se tornado um dos mais populares programas para síntese
sonora do mundo. Constantemente novos recursos são adicionados e
aperfeiçoados, como integração com \sigla{midi}, execução em tempo
real, além de novas facilidades, tanto na construção de orquestras
quanto partituras.

A história do Csound inicia com o programa \prog{Music 4}, escrito no
Bell Telephone Laboratories por Max Mathews no início de 1960. Algumas
adições foram feitas por Godfrey Winham, gerando o \prog{Music 4B}, e
Barry Vercoe, criando em 1968 o \prog{Music 360}. Em 1973 Vercoe criou
o \prog{Music 11} que mais tarde se tornaria o Csound. Em 1991 foram
acrescidos ao Csound \eng{opcodes} para \eng{phase vocoder},
\sigla{fof}, e tipos de dados espectrais. No ano seguinte foram
introduzidos os conversores e as unidades de controle \sigla{midi},
permitindo ao Csound ler arquivos \sigla{midi} como partitura ou
receber dados de teclados musicais externos. Desde então as unidades
de \sigla{midi}, análise, conversão, e tempo-real têm crescido e sido
aperfeiçoadas.

O Csound pode ser considerado atualmente como um dos melhores e mais
poderosos programas para síntese sonora.  Diversos fatores contribuem
para isso:
\begin{description}
\item [multi-plataforma] Existem versões do {Csound} para praticamente
  qualquer plataforma: PC, Mac, Unix, Alpha Digital, Atari, BeOs, etc.
\item [código fonte aberto] Seu código fonte é distribuído através da
  internet, gratuitamente. Com isso diversas pessoas no mundo todo
  auxiliam no seu desenvolvimento, o que contribui para seu rápido
  crescimento. Além disso qualquer um pode modificá-lo e alterá-lo de
  acordo com suas preferências e necessidades.
\item [formato não-proprietário] A entrada de dados é feita através de
  arquivos de texto puro, o que garante uma maior longevidade e
  portabilidade dos dados.
\item [\eng{design} de instrumentos] Ele é ímpar para testar todos os
  processos de síntese sonora, assim como é muito útil na criação de
  unidades para processamento sonoro como \eng{reverbs}, \eng{delays},
  e outros.
\end{description}


\section{Sobre o ministrante} 

Dr. Pedro Kröger é professor adjunto da Escola de Música da UFBA e tem
trabalhado com o Csound há cerca de 12 anos elaborando não apenas
composições, mas também utilizando seu potencial para palestras
didáticas, além de desenvolver tutoriais e ferramentas. Ele contribuiu
com o desenvolvimento da versão ``não-oficial'' do Csound para Linux,
mantida por Nicola Bernardini. Sua tese de doutorado foi centrada no
desenvolvimento de tecnologias e ferramentas para expandir o Csound.

\section{Justificativa}

Disciplinas de graduação envolvendo ``música computacional'' ou
``informática em música'' são cada vez mais comuns em países como
Estados Unidos, Inglaterra, França, Canadá, dentre outros. Já que o
curso de graduação da Escola de Música da UFBA ainda não oferece
disciplinas relacionadas à música eletroacústica faz-se necessário
complementar o conhecimento dos alunos da área de composição com
cursos de extensão.

\section{Objetivos}

Os objetivos do curso são:

\begin{itemize}
\item fornecer os conceitos e técnicas básicas de síntese sonora 
\item apresentar a teoria básica e limitações da representação digital
  do som
\item fornecer os conceitos fundamentais de computação e programação
\item ensinar a linguagem de síntese Csound e como usá-la
\end{itemize}

\section{Conteúdo programático} 

O curso constará de 5 aulas cujo conteúdo programático será como
segue:

\begin{enumerate}
\item Introdução ao Csound: \\
  instalação, configuração, e execução \\
  estrutura da linguagem
\item Introdução ao áudio digital \\
  Mais sobre a Orquestra e a Partitura
  Como estudar orquestra e partituras do Csound
\item Síntese Aditiva \\
  Síntese por modulação: freqüência modulada (FM) e amplitude modulada (AM)
\item Formatos de arquivo de áudio
\item Filtros
\item Síntese granular
\item Filtro de formantes (FOF: \textit{fonctions d'onde formantique})
\item Análise e resíntese:
  introdução a teoria de Fourier \\
  introdução a Convolução \\
\end{enumerate} 

\section{Metodologia}

Cada aula será dividida em três blocos principais; no primeiro serão
apresentados aspectos teóricos e gerais do tópico em questão; no
segundo bloco será visto como esses aspectos podem ser implementados
no Csound; no último bloco serão corrigidos exercícios e dúvidas
individuais esclarecidas.

As aulas serão ministradas no estúdio de música eletroacústica do
Grupo de Computação Musical Genos (GenosLab) e utilizarão de seu
equipamento em beneficio do curso. Um CD com todos os programas
necessários (programas de livre distribuição como o próprio Csound)
será distribuído no início do curso a cada aluno.

Ao final de cada aula serão propostos exercícios e tarefas para
aprofundar o conhecimento adquirido. As tarefas poderão ser feitas em
casa, no caso dos alunos que possuem computador, ou no Laboratório de
Informática da Graduação, onde serão instalados os programas
necessários.

\section{Referências bibliográficas para o curso}
A principal fonte bibliográfica será uma apostila elaborada
especialmente para o curso. Contudo serão recomendadas a leitura de:

\begin{enumerate}
\item Dodge and Jerse, Computer Music: Synthesis, Composition, and
  Performance (2nd Edition) (New York: Schirmer Books, 1997)
\item Boulanger, Richard: The CSound Book (Cambridge: MIT Press, 2000)
 \end{enumerate}

\section{Clientela}
O curso se dirige aos alunos de composição, mas alunos de outras áreas
(inclusive não musicais, como ciência da computação) são bem vindos.
Os alunos devem estar cursando ou ter concluído o curso superior.
  
Os alunos deverão possuir algum conhecimento básico de como utilizar
um sistema operacional (Windows, MacOs, Linux) já que tais
conhecimentos \textbf{não} serão vistos durante o curso.

\section{Duração}

O curso terá 5 aulas de 2 horas.
 
\section{Vagas}
As turmas não devem ultrapassar a quantidade máxima de 13 e mínima de 3
alunos.

\section{Material necessário}
\label{sec:material}

Para o curso serão necessários os seguintes equipamentos:

Para a sala de aula:
\begin{enumerate}
\item Equipamento do estúdio GenosLab
\item Datashow ou projetor similar
\item 13 cadeiras
\item Quadro branco e canetas (LIM)
\end{enumerate}

Para os alunos:

\begin{enumerate}
\item cópia e encadernação da apostila [100-200 páginas]
\item cópia do CD que acompanha a apostila, programas e exemplos
\end{enumerate}

\end{document}
