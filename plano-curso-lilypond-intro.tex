\documentclass[12pt,brazil]{article}
\usepackage[utf8x]{inputenc}
\usepackage{babel}
\usepackage[a4paper,top=2.5cm,left=3cm,right=3cm,bottom=2.5cm]{geometry}
\usepackage{url}

\newcommand{\eng}[1]{\textit{#1}}

\begin{document}

\title{Plano do curso introdutório de Lilypond}
\author{Marcos da Silva Sampaio}
\maketitle
\thispagestyle{empty}

\section{Aula 1 --- Introdução ao curso}

\paragraph{Objetivo}
Apresentar aos alunos informações gerais sobre o Lilypond e uma
introdução à sua sintaxe.

\paragraph{Conteúdo}
\begin{enumerate}
\item Diferenças entre software livre e software proprietário:
  liberdade e desenvolvimento.
\item Página do projeto, documentação, tutoriais e listas de discussão.
\item O papel do tipógrafo musical, problemas com softwares de edição
  de partituras e tipografia automática no Lilypond.
\item Conceito de wysiwyg e diferenças entre Lilypond e outros
  softwares de edição de partituras.
\item Lilypond e formato ly: software e sintaxe.
\item Softwares necessários e \eng{modus operandi} do Lilypond.
\item Exemplos de operações avançadas com o Lilypond.
\item Instalação do Lilypond e do editor Jedit no Linux e Windows.
\item Um pouco de sintaxe: notas, acidentes, duração e
  articulação. (crash course).
\end{enumerate}

\paragraph{Atividades}
\begin{enumerate}
\item Inscrever os alunos na lista genos-users.
\item Ver páginas do projeto e de tutoriais.
\item Experimentar diferentes editores de texto para o Lilypond.
\item Escrever fragmentos simples como \{c d e f\} e processar com o
  Lilypond.
\end{enumerate}

\section{Aula 2 --- Sintaxe 1}
\paragraph{Objetivo}
Apresentar estrutura de um arquivo .ly e comandos gerais básicos.
\paragraph{Conteúdo}
\begin{enumerate}
\item Comentários no .ly.
\item Estrutura geral de um arquivo .ly.
\item Comandos $\backslash$relative, $\backslash$version
\item Elementos musicais: acidentes, ritmo/duração, pausa, silêncio,
  dinâmica, articulação, ligaduras, colchetes, clave, armadura de
  clave, compasso, simultaneidade (acordes) e quiálteras
\end{enumerate}
\paragraph{Atividades}
\begin{enumerate}
\item Escrever fragmentos com elementos musicais variados.
\end{enumerate}

\section{Aula 3 --- Sintaxe 2}
\paragraph{Objetivo}
Apresentação de comandos gerais intermediários.
\paragraph{Conteúdo}
\begin{enumerate}
\item Repetições
\item Elementos musicais: anacruse, barra de compasso, metrônomo,
  dedilhado, tablatura, lyrics, cifra, baixo cifrado.
\item Simultaneidade (polifonia)
\item Textos
\end{enumerate}
\paragraph{Atividades}
\begin{enumerate}
\item Escrever fragmentos com elementos musicais variados.
\item Escrever melodia com lyrics, cifra e baixo cifrado.
\end{enumerate}

\section{Aula 4 --- Variáveis e vários instrumentos}
\paragraph{Objetivo}
Apresentar sintaxe para partituras com vários instrumentos.
\paragraph{Conteúdo}
\begin{enumerate}
\item Estrutura geral do arquivo para conter vários instrumentos.
\item Abstraindo música em variáveis: utilidade e sintaxe.
  \begin{enumerate}
  \item $\backslash$book, $\backslash$score, $\backslash$new staff,
    $\backslash$new voice, nomes dos instrumentos, midi.
  \end{enumerate}
\item Agrupamento de instrumentos em chaves e colchetes.
\end{enumerate}
\paragraph{Atividades}
\begin{enumerate}
\item Escrever fragmento para duo de flautas.
\item Escrever fragmento para coral a 4 vozes em 2 pentagramas.
\item Escrever fragmento para orquestra.
\end{enumerate}

\section{Aula 5 --- Ferramentas extendidas}
\paragraph{Objetivo}
Apresentar programas que dialogam com o Lilypond.
\paragraph{Conteúdo}
\begin{enumerate}
\item Inkscape
\item OOoLilypond
\item Lytex
\item Outras ferramentas do Lilypond.
  \begin{enumerate}
  \item música contemporânea.
  \item educação musical.
  \end{enumerate}
\end{enumerate}
\paragraph{Atividades}
\begin{enumerate}
\item Editar figura gerada pelo Lilypond no Inkscape.
\item Inserir figura do Lilypond no OpenOffice.
\item Criar pequeno arquivo Lytex.
\end{enumerate}

\end{document}
