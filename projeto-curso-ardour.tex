\documentclass{article}
\usepackage[brazil]{babel}
\usepackage[utf8x]{inputenc}
\usepackage{url}
\usepackage[a4paper,top=3cm,bottom=2cm,left=3cm,right=2cm]{geometry}
% \usepackage{achicago}

\title{Estação de Trabalho para Áudio Digital: Introdução ao Software Livre Ardour}

\author{Guilherme Bertissolo}

\begin{document}

\date{\today}

\maketitle
\tableofcontents

\section{Resumo}

O uso do computador nas práticas musicais é cada vez mais comum na
contemporaneidade e a discussão sobre a música eletroacústica nos
âmbitos acadêmicos têm crescido cada vez mais nos últimos
anos. Estação de Trabalho para Áudio Digital: Introdução ao Software
Livre Ardour (ETADISLA) propõe uma abordagem sobre uma poderosa
ferramenta de captação, edição e processamento de áudio digital.  O
objetivo desse curso é possibilitar o uso do Software Ardour para a
composição musical, a partir dos recursos de gravação, edição e
processamento de áudio digital.  Esse curso é dirigido para estudantes
dos cursos de Graduação e Pós-Graduação da Escola de Música da UFBa,
como atividade de Extensão Universitária. A coordenação da atividade
será realizada pelo prof. Dr Pedro Kröger e serão utilizadas as
instalações do GenosLab, Laboratório do Grupo de Pesquisa em
Computação Musical da Escola de Música da UFBa.Guilherme Bertissolo
possui Graduação em Música pela Universidade Estadual do Rio Grande do
Sul (2005). Realizou diversas atividades no Centro de Música
Eletrônica da UFRGS, sob orientação de Eloy Fritsch. Desde 2003
trabalha com música eletroacústica acusmática e mista (suporte fixo e
tempo real). Teve obras estreadas em Porto Alegre, Montenegro,
Curitiba, Salvador e São Paulo. Cursa atualmente o Mestrado em
Composição Musical no PPG-MUS da UFBa.

\section{Introdução}

A Música Eletroacústica é uma manifestação musical amplamente
consolidada nos meios musicais. Basta que olhemos para a quantidade e
qualidade das produções composicionais e publicações que existem a
respeito do tema para nos darmos conta da sua importância. Nos meios
acadêmicos brasileiros e internacionais

Os cursos sobre Música Eletroacústica são oferecidos em maior
quantidade e há mais tempo em instituições estrangeiras, em países
como os Estados Unidos, a Inglaterra, o Canadá e a França.Muitas
universidades já contemplam os seus alunos com componentes
curriculares que discutem essa manifestação. Entretanto, as abordagens
ainda são insuficientes, dada a amplitude e a complexidade do tema.

O curso de Composição da Escola de Música da Universidade Federal da
Bahia sempre foi referência nacional e internacional na formação de
compositores no âmbito universitário. Contudo, uma lacuna no currículo
dos seus cursos de graduação é a falta de um componente curricular que
discuta a música eletroacústica e possibilite a utilização de
ferramentas para a composição musical com recursos eletrônicos.

O GenosLab, Laboratório do Grupo de Pesquisa em Computação Musical
desenvolve pesquisas em teoria e computação musicais. O laboratório
oferece infra-estrutura e equipamentos que possibilitam a prática da
música eletroacústica e a produção e disseminação de conhecimento
nesse campo de estudo.

O curso ETADISLA propõe uma abordagem sobre os recursos de gravação,
edição e processamento de áudio digital, como ferramenta para
composição de música eletroacústica e registro musical em geral.  O
curso possui ao todo dez horas (10hs) de duração e será oferecido a
alunos dos cursos de Graduação e Pós-Graduação da Escola de Música da
UFBa, como atividade de Extensão Universitária, sob orientação do
prof. Dr. Pedro Kröger.


\section{Justificativa}

A necessidade de um curso que tenha como foco as ferramentas para
composição de música eletroacústica e registro musical pode ser medida
tendo-se em vista a emergência cada vez mais comum desse tipo de
proposta em instituições universitárias de ponta, especialmente no
exterior. A partir da demanda curricular dos cursos de Graduação da
Escola de Música supracitada e considerando-se as atividades do
GenosLab é que propomos o curso ETADISLA.


		
\section{Objetivos}

\begin{itemize}
\item Possibilitar o uso do Software Ardour para gravação, edição e
processamento de áudio digital como ferramenta para a composição de
música eletroacústica e registro musical em geral
\end{itemize}

\subsection{Objetivos específicos}

\begin{itemize}
\item Conhecer e operar o Software Ardour;
\item Compreender os princípios básicos de captação e edição de áudio
  digital;
\item Realizar o processamento de áudio através do Software Ardour a
  partir de suas próprias ferramentas e das oferecidas através de
  plug-ins LADSPA;
\item Realizar exercícios práticos relacionados aos temas abordados.
\end{itemize}

\section{Conteúdo programático}

O curso constará de 5 aulas cujo conteúdo programático será como segue:
\begin{itemize}
\item Introdução às Estaçõe de Trabalho para Áudio Digital;
\item Princípios de Acústica Musical;
\item Noções básicas de informática em música;
\item Processo de digitalização do áudio;
\item Operações básicas do Software Ardour;
\item Técnicas de gravação);
\item Edição e processamento do áudio digital;
\item Mixagem;
\item Finalização de arquivos de áudio.
\end{itemize}


\section{Metodologia}

As aulas possuirão basicamente dois formatos. O primeiro deles
privilegiará uma abordagem mais expositiva, enquanto que o outro é
correspondente às aulas predominantemente práticas.  Nas aulas com
formato expositivo teórico-prático serão abordadas noções básicas
sobre a acústica musical e os princípios básicos de gravação, de
maneira a possibilitar o uso mais adequado dos equipamentos
disponíveis. Há que se salientar que esse tipo de formato mais
expositivo dado às aulas não exclui uma intensa participação dos
alunos e um fomento ao debate sobre os tópicos abordados. Muito pelo
contrário, a busca pela relativização dos conceitos e pela crítica às
definições abordadas será constantemente incentivada no decorrer das
sessões.

O outro formato privilegiará uma abordagem prática. Dessa forma, os
alunos poderão entrar em contato direto com o software, de maneira a
terem uma participação ativa. Serão realizadas gravações e atividades
de edição, processamento, mixagem, etc. Algumas exercícios serão
realizados em aula, outros em casa pelos alunos que tiverem
computadores disponíveis, ou, para os que não tiverem, em outros
horários a serem combinados no primeiro dia de aula.

Os recursos utilizados nas aulas serão os computadores e equipamentos
do GenosLab, quadro branco e projetor multimídia.


\section{Das Aulas}

As aulas acontecerão no laboratório do GenosLab, na Escola de Música
da UFBa, diariamente durante uma semana. Como a carga horária diária
será de duas horas, o curso terá um total de dez horas aula.

\section{Das Vagas}

A turma deverá ter entre 3 e 12 alunos, que deverão fazer inscrição
prévia.

\nocite{roads96:_comput_music_tutor,gibson02:_artis_pro_series,iazzetad.:_tutor_de_e,boulanger00:_csoun_book,dodge97:_comput_music,menezes03:_music_em_palav_e_sons,menezes96,menezes06:_maxim,moore90:_elemen_of_comput_music,schaeffer93:_tratad_dos_objet_music,d.:_manual_ardour}

\bibliographystyle{plain}
% \bibliographystyle{achicago}
\bibliography{mestrado-guilherme}


\end{document}

