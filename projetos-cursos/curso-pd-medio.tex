\documentclass{article}
\usepackage[brazil]{babel}
\usepackage[utf8x]{inputenc}


\title{Curso intermediário de Pure data }

\author{Cristiano Figueiró }

\begin{document}

\date{\today}

\maketitle
%%\tableofcontents
\section{ Resumo}


Pd trabalha com, digamos, "dados puros". Essa é uma das diferenças entre ele e o Max/MSP, seu concorrente comercial. Por trabalhar assim, seu âmbito de realizações acaba sendo potencialmente muito maior que o do Max/MSP. Os comandos do Pd são muito próximos aos da linguagem de programação C, o que pode facilitar o seu aprendizado para quem já possui um conhecimento prévio dessa linguagem.

Pd foi criado para explorar idéias de como promover e permitir que dados possam ser tratados de maneira mais aberta, facilitando acesso e interligação entre aplicações de áudio, MIDI, gráficas e vídeo, dentre outras possibilidades.

Escrevendo objetos ("externals") ou módulos ("patches", "abstractions") consegue-se ampliar as funcionalidades do Pd. Os trabalhos de muitos desenvolvedores estão disponíveis como parte do pacote básico do Pd e a comunidade de desenvolvedores cresce muito rapidamente.

Desenvolvimentos recentes incluem projetos de sistemas para construção de ambientes de performances; bibliotecas de objetos para "physical modeling"; bibliotecas de objetos para geração e processamento de vídeo em tempo real, etc...

Nesse curso iremos aprofundar os conceitos já trabalhados em outros dois cursos anteriores realizados com sucesso. Entre os tópicos a ser abordados destacamos o aprofundamento da análise de áudio em tempo-real, conexão do Pd com Arduino e computação física, composição algorítmica, conexão com outros programas e composição colaborativa através da internet (NetPd).



\section{Introdução}


       O Pure data (Pd) pode ser considerado atualmente como um dos melhores e mais
poderosos programas para composição de música interativa aliando as possibilidades
computacionais de síntese sonora e processamento de áudio a um ambiente gráfico de
programação voltado à interação sonora em tempo-real. Se trata de um programa gráfico
com interface amigável, multi-plataforma, com o código fonte aberto, e com formato não
proprietário. Ele é a terceira maior ramificação da família de programas que usam
linguagem de programação modular, conhecida como Max (Max/MSP, jMax, etc.)
originalmente desenvolvida por Miller Puckette (IRCAM). O núcleo do Pd é escrito e
mantido por Miller Puckette e adicionado ao trabalho de muitos outros desenvolvedores.
      

 O objetivo é oferecer um curso intermediário desse programa onde além de
demonstrar as possibilidades e limitações dessa ferramenta o participante adquira uma
autonomia de estudo mais aprofundado do programa. Além disso, entender os conceitos e
técnicas básicas de síntese sonora na prática do programa, entender as possibilidades e
limitações da interação em tempo-real entre músico e computador, aprender os conceitos
de programação em Pd e como usá-lo num projeto composicional.

O Pd é uma linguagem gráfica orientada ao objeto e desenvolvido por uma comunidade de programadores.
O núcleo da linguagem é desenvolvida e mantida por seu criador Miller Puckette, enquanto as outras bibliotecas (conexões com outras linguagens e utilitários diversos) ficam a cargo da comunidade. Apesar de já existir um protocolo de desenvolvimento desses objetos externos, cada desenvolvedor agrega novos estilos e recursos
 de compilação desses objetos. Como o Pd é distribuído no nível do código fica relativamente complexo administrar a compilação de todas essas bibliotecas. Nesse curso iremos tratar sobre esses problemas de compilação de objetos externos o que por fim possibilita uma maior usabilidade e poder de interação com outros programas e linguagens.
      

 O curso terá 5 aulas de 2 horas em dias consecutivos (compreendendo uma semana
letiva inteira). O curso se dirige aos alunos de composição musical, mas alunos de outras
áreas (inclusive não musicais, como ciência da computação) são bem vindos. Os alunos
devem estar cursando ou ter concluído o curso superior. O conhecimento básico de como
utilizar um sistema operacional (Windows, MacOs, Linux) é requerido.
      

 Cristiano Figueiró tem trabalhado com o Pd e Max/MSP há quatro anos
elaborando composições e artigos científicos sobre os paradigmas composicionais desses
programas. Sua dissertação de mestrado está centrada no desenvolvimento de estratégias
composicionais para a composição de peças eletroacústicas mistas envolvendo a
interação entre músicos e computador em tempo real.


\section{Justificativa}
\label{sec:justificativa}
           Disciplinas de graduação envolvendo música computacional ou informática em
música são cada vez mais comuns em países como Estados Unidos, Inglaterra, França,
Canadá, dentre outros. Já que o curso de graduação da Escola de Música da UFBA ainda
não oferece disciplinas relacionadas à música computacional faz-se necessário
complementar o conhecimento dos alunos da área de composição com cursos de extensão.
	
\section {Objetivos}

Os objetivos do curso são:\\
- ensinar a linguagem de programação gráfica Pd e como usá-la;\\
- fornecer os conceitos básicos de implementação das principais técnicas de síntese sonora
e processamento e análise de áudio em Pd;\\
- fornecer os conceitos fundamentais de interação e exemplos de projetos composicionais
envolvendo interação entre músicos e computadores;\\

\section {Conteúdo programático}

O curso constará de 5 aulas cujo conteúdo programático será como segue:\\
1.  Revisão da introdução ao Pd:
instalação, configuração e execução, estrutura da linguagem, operações matemáticas básicas
e fluxo de dados;\\
2. Protocolo MIDI, envelopes e automações, composição algorítmica e conexão com outros programas;\\
3. Implementação de técnicas clássicas de síntese sonora: Síntese Aditiva,
síntese por modulação: freqüência modulada (FM) e amplitude modulada (AM),
sampleamento, filtros e delay;\\
4. Encapsulamento e sub-patchs, conexão do Pd com Arduino;\\
5. Introdução a FFT (Fast Fourier Transform) em Pd, análise de áudio de entrada em
tempo-real (objeto fiddle~) e e composição colaborativa através da internet (NetPd);\\

\section {Metodologia}

       Cada aula será dividida em três blocos principais; no primeiro serão apresentados
aspectos teóricos e gerais do tópico em questão; no segundo bloco será visto como esses
aspectos podem ser implementados no Pd; no último bloco serão corrigidos exercícios e
dúvidas individuais esclarecidas.
       As aulas serão ministradas no estúdio de música eletroacústica Genos e utilizarão de
seu equipamento em benefício do curso. Um CD com todos os programas e exemplos
necessários (Pd, alguns objetos externos, tutoriais e exemplos que serão abordados em aula)
será distribuído no início do curso a cada aluno. Ao final de cada aula serão propostos
exercícios e tarefas para aprofundar o conhecimento adquirido. As tarefas poderão ser feitas
em casa, no caso dos alunos que possuem computador, ou no próprio Genos, em horários
marcados        na      primeira     aula.    Um      fórum       na       internet     (em
http://wiki.genos.mus.br/cursopd/forum) foi criado onde os alunos podem tirar dúvidas e
trocar sugestões entre si.

\section {Vagas}

As turmas não devem ultrapassar a quantidade máxima de 13 e mínima de 3 alunos.

\section {Referências Bibliográficas}
1. Farnell, A. , Composition in puredata 1. Acessado em 25 de maio de 2007, em:\\
http://www.obiwannabe.co.uk/html/music/musictuts/compose1/com
position1/
composition1.html.\\
2. Puckette, M. A divide between ’compositional’ and ’performative’ aspects of pd. In
Proceedings of the First Pd convention (2004).\\
3. Roads, C. The Computer Music Tutorial. MIT Press, Massachussets (1996).\\
4. Winkler, T. Composing Interactive Music - Techniques and ideas using Max. MIT Press,
Massachussets (1993).

\section { Clientela}
O curso se dirige aos alunos de composição, mas alunos de outras áreas (inclusive não
musicais, como ciência da computação) são bem vindos. Os alunos devem estar cursando
ou ter concluído o curso superior. Os alunos deverão possuir algum conhecimento básico de
como utilizar um sistema operacional (Windows, MacOs, Linux) já que tais conhecimentos
não serão vistos durante o curso.
\section {Duração} 
O curso terá 5 aulas de 2 horas.

\section {Vagas} 
As turmas não devem ultrapassar a quantidade máxima de 13 e mínima de 3 alunos.
\section {Material necessário}  
Para o curso serão necessários os seguintes equipamentos:\\
Para a sala de aula:\\
1. Equipamento do estúdio Genos\\
2. Datashow\\
3. 13 cadeiras\\
4. Quadro branco e canetas (Genos)\\

\end{document}

%%%%C-c C-c para compilar e visualizar

