\documentclass[12pt,brazil]{article}
\usepackage[utf8x]{inputenc}
\usepackage{babel}
\usepackage[top=25mm,bottom=25mm,left=25mm,right=25mm]{geometry}
\usepackage{url}

\newcommand{\sigla}[1]{\textsc{#1}}
\newcommand{\eng}[1]{\textit{#1}}
\newcommand{\prog}[1]{\textsf{#1}}

\begin{document}
\selectlanguage{brazil}

\title{Escrevendo documentos acadêmicos com Linux, \LaTeX{} e Emacs}
\author{Marcos di Silva}
\maketitle

\thispagestyle{empty}

\section{Introdução}

\LaTeX{} é um sistema de tipografia de alta qualidade otimizado para
documentos estruturados e acadêmicos. Ele permite a criação fácil de
artigos, livros, relatórios, cartas, e apresentações. Esse curso vai
abordar os principais recursos para produzir documentos de maneira
profissional e produtiva, incluindo o gerenciamento de bibliografias
com BibTeX, uso de templates diversos como do SBC, ACM, e ANPPOM, uso
de figuras, referência cruzadas, código fonte, tabelas e fórmulas
matemáticas.

No \LaTeX{} o documento é descrito usando marcações. As principais
vantagens do \LaTeX{} sobre outros sistemas são:

\begin{itemize}
\item Layout Lógico e separação de conteúdo e layout
\item Resultado Tipográfico Superior
\item Portabilidade
\item Estabilidade
\item Disponibilidade
\item Menores necessidades de hardware
\item Formato dos Arquivos em texto puro
\item Longevidade dos documentos
\item Geração de Referências Bibliográficas
\end{itemize}

O Linux é um sistema operacional (como Mac ou Windows) e o Emacs é um
editor de texto puro altamente configurável. Estas duas ferramentas
dispõe de vários recursos para o uso do \LaTeX.

\section{Justificativa}

Este curso está sendo oferecido para atender uma demanda bem
específica de alguns alunos do Programa de pós-graduação em Música da
UFBA que já conhecem o \LaTeX{} e têm usado este sistema em seus
projetos de tese/dissertação, artigos e relatórios. Estes alunos têm
requisitado um curso avançado de \LaTeX{} para poderem produzir seus
textos de forma mais produtiva.

\section{Objetivos}

Os objetivos do curso são:

\begin{itemize}
\item revisar conceitos e técnicas básicas do \LaTeX, e elementos de
  texto como seções, fontes, ambientes para itemização, enumeração,
  citações, figuras e tabelas
\item apresentar como abstrair comandos com o \LaTeX
\item apresentar como usar estilos diferentes para um mesmo documento
\item apresentar como gerenciar referências cruzadas e bibliografias
  em documentos acadêmicos
\item apresentar como usar o Emacs e auctex para trabalhar com \LaTeX
\end{itemize}

\section{Conteúdo programático} 

O curso constará de 5 aulas cujo conteúdo programático será como
segue:

\begin{enumerate}
\item revisão do básico de \LaTeX: o que é e para que serve, conceitos
  e técnicas
\item revisão de comandos do LaTeX e elementos de texto
\item configuração de variáveis de ambiente e encoding
\item caracteres especiais
\item comandos para fontes
\item inclusão de gráficos, tabelas e figuras
\item criação de comandos
\item referências cruzadas
\item bibliografia com BibTeX
\item usando estilos \LaTeX{} do PPGMUS, SBCM e ANPPOM
\item modificação do layout
\item introdução ao Emacs: o que é, para que serve
\item configuração mínima do Emacs
\item uso do Auctex
\end{enumerate} 

\section{Metodologia}

Cada aula será dividida em dois blocos principais. No primeiro serão
apresentados aspectos teóricos e gerais do tópico em questão. No
segundo bloco os alunos praticarão os tópicos abordados. Durante as
aulas serão propostos exercícios e tarefas para aprofundar o
conhecimento adquirido. As aulas serão ministradas no GenosLab,
laboratório do grupo de pesquisa Genos, e cada aluno deverá usar seu
próprio laptop (vide Clientela)

\section{Referências bibliográficas para o curso}

A principal fonte bibliográfica será a apostila disponível na internet
``Introdução ao LATEX2''\footnote{Disponível em
  \url{www.ctan.org/tex-archive/info/lshort/portuguese/ptlshort.pdf}.}. O
\LaTeX{} e o Emacs têm excelente documentação e guias nas página dos
respectivos projetos\footnote{\LaTeX{} disponível em
  \url{http://www.latex-project.org/}, e Emacs em
  \url{http://www.gnu.org/software/emacs/}.},

\section{Clientela}

O curso se destina aos alunos do Programa de pós-graduação em Música
da UFBA que já usam \LaTeX{} e Linux, e que têm um laptop para fazer o
curso. Todos os alunos que requisitaram o curso têm laptop. Fazer o
curso no próprio computador dinamiza o aprendizado, pois os possíveis
problemas de instalação e configuração das ferramentas podem ser
vistos \textit{in loco}.

\section{Duração}

O curso terá 5 aulas de 2 horas.
 
\section{Vagas}

Este curso deve ter um número de 2 a 5 alunos.

\section{Material necessário}
\label{sec:material}

Para o curso serão necessários os seguintes equipamentos:

Para a sala de aula:
\begin{enumerate}
\item Datashow ou projetor similar
\item Quadro branco e canetas
\end{enumerate}

Para o aluno:
\begin{enumerate}
\item Laptop com Linux instalado e funcionando
\end{enumerate}

\end{document}
