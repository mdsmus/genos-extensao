\documentclass[12pt,brazil]{article}
\usepackage[utf8x]{inputenc}
\usepackage{babel}
\usepackage{url}
\usepackage[top=25mm,bottom=25mm,left=25mm,right=25mm]{geometry}

\newcommand{\sigla}[1]{\textsc{#1}}
\newcommand{\eng}[1]{\textit{#1}}
\newcommand{\prog}[1]{\textsf{#1}}

\begin{document}
\selectlanguage{brazil}

\title{Curso de capacitação em ferramentas computacionais}
\author{Pedro Kröger}
\maketitle

\begin{abstract}
  O desenvolvimento de grupos de pesquisa multi-disciplinares é
  dificultado pela necessidade dos participantes terem que dominar
  conhecimentos e ferramentas em mais de uma área. No caso específico
  do Genos existem algumas ferramentas computacionais que são
  utilizadas em pesquisas, aulas e cursos. Contudo, essas ferramentas
  não são ensinadas em cursos de graduação em música e muitas vezes
  nem em cursos de computação. Nesse curso serão abordadas as
  ferramentas computacionais mais utilizadas no Genos: Emacs, Git, e
  Gnu Make. Além disso serão abordadas técnicas de trabalho com
  emuladores de terminal e sincronia de arquivos e acesso remoto com
  ssh e unison.
\end{abstract}

\thispagestyle{empty}

\section{Introdução}

O desenvolvimento de grupos de pesquisa multi-disciplinares é
dificultado pela necessidade dos participantes terem que dominar
conhecimentos e ferramentas em mais de uma área. No caso específico do
Genos existem algumas ferramentas computacionais que são utilizadas em
pesquisas, aulas e cursos. Contudo, essas ferramentas não são
ensinadas em cursos de graduação em música e muitas vezes nem em
cursos de computação. Nesse curso serão abordadas as ferramentas
computacionais mais utilizadas no Genos: Emacs, Git, e Gnu Make. Além
disso serão abordadas técnicas de trabalho com emuladores de terminal
e sincronia de arquivos e acesso remoto com ssh e unison.

Emacs é um dos editores de texto mais poderosos que existem, podendo
ser customizado para executar diferentes tarefas. Como ele tem uma
curva de aprendizado grande alguns usuários acabam não percebendo seu
potencial. Com ele é possível criar um ambiente integrado de
desenvolvimento completo e produtivo para a programação com recursos
de ``alto nível'' para processar textos como ``apague o parágrafo
anterior'' ou ``á para o quarto parágrafo''; recursos para criar
documentos com marcação como LeTeX e HTML; integração com ferramentas
de controle de versão como RCS, CVS, subversion, e GNU Arch; recursos
específicos para linguagens de programação como sintaxe colorida,
execução e compilação automática, depuramento, navegador de classes,
dentre outros; facilidades como auto-complementação, TAGS, e
abreviações; recursos de documentação do Emacs, como ler paginas de
manual do unix; ferramentas de internet como email, ftp, e
navegadores; recursos avançados de procura e substituição; dentre
outros.

O Git é um programa para controle de versão distribuído desenvolvido
por Linus Torvalds (o criador do Linux) e mantido por Junio C Hamano.
Com ele é possível manter versões diferentes do mesmo documento
permitindo acessar versões anteriores, criar ramos locais e remotos,
controlar o histórico de mudanças, etc.

O Gnu Make é um sistema de construção que pode ser usado para
automatizar a construção (\textit{build}) de programas de computador,
documento de textos escritos em \LaTeX, composições em LilyPond,
dentre outros.

\section{Sobre os ministrantes} 

Dr. Pedro Kröger é professor adjunto da Escola de Música da UFBA e tem
trabalhado com ferramentas computacionais há cerca de 12 anos com uso
aplicado em composição, computação, pesquisa e ensino.

Alexandre Passos é estudante de Ciência da Computação na UFBA, membro
do Genos e bolsista do PIBIC.

\section{Justificativa}

Como não existem disciplinas na graduação que abordem ferramentas como
Emacs, Git, e Gnu Make, faz-se necessário complementar o conhecimento
dos alunos com interesse em música computacional com cursos de
extensão.

\section{Objetivos}

Os objetivos do curso são:

\begin{itemize}
\item fornecer as principais técnicas e conceitos intermediários e
  avançados de edição com emacs, desenvolvimento com o Git e
  construção com Gnu Make
\item apresentar técnicas diversas para sincronia de arquivos e acesso
  remoto com ssh e unison
\item apresentar técnicas avançadas de uso de emulador de terminal
\end{itemize}

\section{Conteúdo programático} 

O curso constará de 15 aulas cujo conteúdo programático será como
segue:

\begin{itemize}
\item Emacs
  \begin{itemize}
  \item customização do emacs (\textit{custom} e técnicas para o .emacs)
  \item edição básica
  \item buscar e substituir avançado
  \item buffers, janelas, frames
  \item dired
  \item eshell
  \item documentação com info
  \item automação com macros
  \item modo outline
  \item programação em C e gdb
  \item controle de versão (git, svn, cvs)
  \item expandindo o emacs com elisp
  \end{itemize} 
\item Git
  \begin{itemize}
  \item trabalhando com branches
  \item automação com hooks
  \item criando sub-projetos
  \item introdução aos \textit{internals}
  \item explorando histórico
  \item \textit{workflow} de desenvolvimento
  \item criando repertórios remotos
  \end{itemize}
\item Gnu Make
  \begin{itemize}
  \item escrevendo regras
  \item regras implícitas
  \item comandos nas regras
  \item uso de variáveis
  \item uso de condicionais
  \item funções para transformar texto
  \end{itemize}
\item Emulador de terminal
\item Acesso remoto com ssh e unison
\end{itemize}
  
\section{Metodologia}

As aulas serão ministradas no laboratório de música eletroacústica do
Grupo de Computação Musical Genos (GenosLab) e utilizarão de seu
equipamento em beneficio do curso.

As aulas terão uma parte expositiva e uma prática, onde os alunos
praticarão os assuntos em computadores.

Ao final de cada aula serão propostos exercícios e tarefas para
aprofundar o conhecimento adquirido. As tarefas poderão ser feitas em
casa, no caso dos alunos que possuem computador, ou no Laboratório de
Informática da Graduação, onde serão instalados os programas
necessários.

\section{Referências bibliográficas para o curso}

As principais fontes bibliográficas serão os manuais dos programas
disponíveis na internet:

\begin{enumerate}
\item Manual do Git disponível em \url{http://www.kernel.org/pub/software/scm/git/docs/user-manual.html}
\item Manual do Emacs disponível em \url{http://www.gnu.org/software/emacs/manual/}
\item Manual do Gnu Make disponível em \url{http://www.gnu.org/software/make/manual/make.html}
\end{enumerate}

\section{Clientela}

O curso se dirige aos membro do Genos, mas alunos de composição e de
outras áreas (inclusive não musicais, como ciência da computação) são
bem vindos. Os alunos devem estar cursando ou ter concluído o curso
superior e possuir algum conhecimento básico dos programas abordados
no curso.

\section{Duração}

O curso terá 15 aulas de 2 horas.
 
\section{Vagas}
As turmas não devem ultrapassar a quantidade máxima de 13 e mínima de 3
alunos.

\section{Material necessário}
\label{sec:material}

Para o curso serão necessários os seguintes equipamentos:

Para a sala de aula:
\begin{enumerate}
\item Equipamento do estúdio GenosLab
\item Datashow ou projetor similar
\item 13 cadeiras
\item Quadro branco e canetas (LIM)
\end{enumerate}

\end{document}
