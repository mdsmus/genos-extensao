\documentclass{article}
\usepackage[brazil]{babel}
\usepackage[utf8x]{inputenc}
\usepackage{url}
\usepackage[a4paper,top=3cm,bottom=2cm,left=3cm,right=2cm]{geometry}
% \usepackage{achicago}

\title{Curso de introdução ao Áudio digital}

\author{Cristiano Figueiró e Guilherme Bertissolo}

\begin{document}

\date{\today}

\maketitle
\tableofcontents

\section{Resumo}

O Áudio Digital (AD) é um dos meios de comunicação mais usados e
aceitos atualmente como padrão internacional. Podemos ver o AD sendo
usado nas emissoras de Tv, rádios, internet, Cd's, Dvd's, para
citarmos apenas alguns exemplos. Além disso, a atividade prática da
maioria dos músicos contemporâneas passa pela manipulação de áudio
digital, seja pelo simples registro de uma prática musical
instrumental/vocal, seja na composição de música eletroacústica.
 
O objetivo desse curso é oferecer uma introdução aos princípios do
AD,suas possibilidades e limitações. A idéia é que os participantes
adquiram uma autonomia para um estudo mais aprofundado sobre o
tema. Além disso, entender os conceitos e técnicas básicas de
manipulação, edição e captação de AD, de maneira a introduzir e
possibilitar o desenvolvimento do conhecimento em computação
musical. Esse curso terá seus desdobramentos inclusive nos outros
cursos que serão oferecidos pelo GenosLab durante o ano corrente.

O curso terá um total de cinco aulas de duas horas em dias
consecutivos, compreendendo uma semana letiva inteira (totalizando dez
horas/aula). O curso se dirige aos alunos dos cursos de graduação e
pós-graduação em música da UFBa, mas alunos de outras cursos também
poderão participar. O conhecimento básico de como utilizar um sistema
operacional (Windows, MacOs, Linux) é requerido.

Cristiano Figueiró tem trabalhado com o Pd e Max/MSP há cinco anos
elaborando composições e artigos científicos sobre os paradigmas
composicionais desses programas. Sua dissertação de mestrado está
centrada no desenvolvimento de estratégias composicionais para a
composição de peças eletroacústicas mistas envolvendo a interação
entre músicos e computador em tempo real. Em 2006 trabalhou como
compositor e produtor-assistente da gravação do Cd "In Itinere" do
grupo de música eletroacústica da UFG. Atualmente cursa o Doutorado em
Composição Musical na Escola de Música da UFBa.

Guilherme Bertissolo possui Graduação em Música pela Universidade
Estadual do Rio Grande do Sul (2005). Realizou diversas atividades no
Centro de Música Eletrônica da UFRGS, sob orientação de Eloy
Fritsch. Desde 2003 trabalha com música eletroacústica acusmática e
mista (suporte fixo e tempo real). Teve obras estreadas em Porto
Alegre, Montenegro, Curitiba, Salvador e São Paulo. Cursa atualmente o
Mestrado em Composição Musical no PPG-MUS da UFBa.


\section{Introdução}

O AD, além de um dos principais meios de comunicação atuais, é um
grande aliado na prática diária do músico. O AD permite, entre outras
coisas, gravar, fazer arranjos, compor, analisar e extrair informações
de gravações, tudo isso usando apenas um computador pessoal.

A história do AD inicia com os experimentos de Max Matthews em 1957,
na Bell Laboratories em Nova York. Foi lá onde, primeira vez,
conseguiu-se sintetizar um som de 17 segundos através de um
computador. O próximo passo foi, por um lado desenvolver as técnicas
de síntese sonora e, por outro, desenvolver um mecanismo de amostragem
do som que pudesse converter o áudio analógico para AD sem
perdas. Comercialmente o AD começou a ser usado no fim dos anos 70 na
indústria fonográfica.

Atualmente o AD é usado em todas as fases de uma produção
musical comercial (pré-produção, gravação, mixagem, masterização e
distribuição). Além de ser peça fundamental na telefonia (VoIp e
celulares), televisão, internet, jogos, etc.

Nesse curso vão ser utilizados softwares livres rodando sobre uma
plataforma livre (Linux), porém alunos que não usem Linux podem se
inscrever pois poderão utilizar os conhecimentos em qualquer
plataforma.  Os programas utilizados serão: Audacity (edição básica de
áudio), Rosegarden (seqüenciamento de MIDI e processamento de áudio) e
Ardour (Digital Áudio WorkStation ou Estação de Trabalho para Áudio Digital).

\section{Justificativa}

Esse curso surge de uma constatação de que a grande parte dos alunos
dos cursos de Graduação e Pós-Graduação da Escola de Música da UFBa já
trabalha com AD e, comumentemente, sem orientação adequada e sem uma
perspectiva profissional.

Disciplinas de graduação envolvendo música computacional ou
informática em música são cada vez mais comuns em países como Estados
Unidos, Inglaterra, França, Canadá, dentre outros. Já que os cursos de
Graduação e Pós-Graduação da Escola de Música da UFBA ainda não
oferecem disciplinas relacionadas à música computacional, faz-se
necessário complementar o conhecimento dos alunos nesse sentido a
partir de uma atividade de extensão.
		
\section{Objetivos}

\begin{itemize}
\item Compreender os conceitos básicos sobre o AD;
\item Entender as operações básicas de softwares livres para
  manipulação de AD;
\item Oferecer uma mudança de perspectiva do aluno de "usuário" de programas de
  computador para a um patamar de artista criador de AD.
\end{itemize}
\section{Conteúdo programático}

O curso constará de 5 aulas cujo conteúdo programático será como segue:
\begin{itemize}
\item Princípios de Acústica;
\item Noções básicas de informática em música;
\item Áudio analógico;
\item Processo de digitalização do áudio;
\item Formatos, compactação, mídias;
\item Técnicas de gravação (melhor aproveitamento de acústica de salas e microfones);
\item Programas de gravação e produção em software livre: Audacity, Ardour, Rosegarden;
\item Mixagem e remixagem;
\item Masterização e pós-produção;
\end{itemize}

\section{Metodologia}

Cada aula será dividida em dois blocos principais; no primeiro serão
apresentados aspectos teóricos e gerais do tópico em questão; no
segundo bloco serão corrigidos exercícios e dúvidas individuais.

As aulas serão ministradas no estúdio de música eletroacústica Genos e
utilizarão de seu equipamento em benefício do curso. Ao final de cada
aula serão propostos exercícios e tarefas para aprofundar o
conhecimento adquirido. As tarefas poderão ser feitas em casa, no caso
dos alunos que possuem computador, ou no próprio Genos, em horários
marcados na primeira aula.  Um fórum na internet (em
\url{http://wiki.genos.mus.br/cursoaudio}) foi criado onde os alunos
podem tirar dúvidas e pesquisar em links externos.

\section{Vagas}

A turma deverá ter entre 3 e 12 alunos, que deverão fazer inscrição prévia.

\nocite{roads96:_comput_music_tutor,gibson02:_artis_pro_series,iazzetad.:_tutor_de_e,boulanger00:_csoun_book,dodge97:_comput_music,menezes03:_music_em_palav_e_sons,menezes96,menezes06:_maxim,moore90:_elemen_of_comput_music,schaeffer93:_tratad_dos_objet_music}

\bibliographystyle{plain}
% \bibliographystyle{achicago}
\bibliography{mestrado-guilherme}

\end{document}
