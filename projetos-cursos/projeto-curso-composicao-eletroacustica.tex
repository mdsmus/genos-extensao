\documentclass[12pt]{article}
\usepackage[brazil]{babel}
\usepackage[utf8x]{inputenc}
\usepackage{url}
\usepackage[a4paper,top=3cm,bottom=2cm,left=3cm,right=2cm]{geometry}
\usepackage{graphicx}
\usepackage{achicago}
\usepackage{setspace}

\title{Curso de Introdução à Composição Eletroacústica}

\author{Guilherme Bertissolo \footnote{Guilherme Bertissolo possui Graduação em Música pela Universidade
Estadual do Rio Grande do Sul (2005). Realizou diversas atividades no
Centro de Música Eletrônica da UFRGS, sob orientação de Eloy
Fritsch. Desde 2003 trabalha com música eletroacústica acusmática e
mista (suporte fixo e tempo real). Teve obras estreadas em Porto
Alegre, Montenegro, Curitiba, Salvador e São Paulo. Cursa atualmente o
Mestrado em Composição Musical no PPG-MUS da UFBa e é membro do Grupo
de Pesquisa Genos.}}

\date{ }

\begin{document}

\maketitle
\tableofcontents

\begin{abstract}

  O uso do computador nas práticas musicais é cada vez mais comum na
  contemporaneidade e a discussão sobre a música eletroacústica nos
  âmbitos acadêmicos tem crescido cada vez mais nos últimos
  anos. \textit{Introdução à Composição Eletroacústica} propõe uma
  abordagem sobre a prática da composição em estúdio, utilizando
  recursos tecnológicos para sua produção. O curso é dirigido para
  estudantes dos cursos de Graduação e Pós-Graduação da Escola de
  Música da UFBa, como atividade de Extensão Universitária. A
  coordenação será realizada pelo prof. Dr Pedro Kröger e serão
  utilizadas as instalações do GenosLab, Laboratório do Grupo de
  Pesquisa em Computação Musical da Escola de Música da UFBa.

\end{abstract}

\section{Introdução}

A Música Eletroacústica é uma manifestação musical amplamente
consolidada nos meios musicais. Basta que olhemos para a quantidade e
qualidade das produções composicionais e publicações que existem a
respeito do tema para nos darmos conta da sua importância \footnote{No
  Brasil temos já diversos centros em música eletroacústica, como o
  Centro de Música Eletrônica, da UFRGS, o Estúdio da Glória, no Rio
  de Janeiro. Diversos discos e livros têm sido publicados no Brasil e
  no exterior, fundamentando essa prática cada vez mais
  difundida.}.

Os cursos sobre Música Eletroacústica são oferecidos em maior
quantidade e há mais tempo em instituições estrangeiras, em países
como os Estados Unidos, a Inglaterra, o Canadá e a França.Muitas
universidades já contemplam os seus alunos com componentes
curriculares que discutem essa manifestação. Entretanto, as abordagens
ainda são insuficientes, dada a amplitude e a complexidade do tema.

O curso de Composição da Escola de Música da Universidade Federal da
Bahia sempre foi referência nacional e internacional na formação de
compositores no âmbito universitário. Contudo, uma lacuna no currículo
dos seus cursos de graduação é a falta de um componente curricular que
discuta a música eletroacústica e possibilite a utilização de
ferramentas para a composição musical com recursos eletrônicos.

O GenosLab, Laboratório do Grupo de Pesquisa em Computação Musical
desenvolve pesquisas em teoria e computação musicais. O laboratório
oferece infra-estrutura e equipamentos que possibilitam a prática da
música eletroacústica e a produção e disseminação de conhecimento
nesse campo de estudo.

O curso \textit{Introdução à Música Eletroacústica} propõe uma
abordagem sobre a composição musical em estúdio, lançando mão de
recursos de síntese sonora, gravação, edição e processamento de áudio
digital no contexto da música eletroacústica. O curso possui ao todo
dez horas (10hs) de duração e será oferecido a alunos dos cursos de
Graduação e Pós-Graduação da Escola de Música da UFBa, como atividade
de Extensão Universitária, sob orientação do prof. Dr. Pedro Kröger.
O curso terá um total de cinco aulas de duas horas em dias
consecutivos, compreendendo uma semana letiva inteira (totalizando dez
horas/aula). É dirigido aos alunos dos cursos de graduação e
pós-graduação em música da UFBa, mas alunos de outras cursos também
poderão participar.

\section{Justificativa}

Esse curso surge de uma constatação de que a grande parte dos alunos
dos cursos de Graduação e Pós-Graduação da Escola de Música da UFBa já
trabalha com recursos eletrônicos/eletroacústica para a sua prática
musical e, comumentemente, sem orientação adequada e sem uma
perspectiva profissional. Disciplinas de graduação envolvendo música
computacional ou informática em música são cada vez mais comuns em
países como Estados Unidos, Inglaterra, França, Canadá, dentre
outros. Já que os cursos de Graduação e Pós-Graduação da Escola de
Música da UFBA ainda não oferecem disciplinas relacionadas à música
computacional, faz-se necessário complementar o conhecimento dos
alunos nesse sentido a partir de uma atividade de extensão.

A necessidade de um curso que tenha como foco a composição de música
eletroacústica pode ser medida tendo-se em vista a emergência cada vez
mais comum desse tipo de proposta em instituições universitárias de
ponta, especialmente no exterior. A partir da demanda curricular dos
cursos de Graduação da Escola de Música supracitada e considerando-se
as atividades do GenosLab é que propomos o curso Introdução à
Composição eletroacústica.

\section{Objetivos}

\begin{itemize}
\item Compreender os fundamentos da música eletroacústica;
\item Realizar pequenos estudos de composição em estúdio;
\item Entender as especificidades dessa prática musical;
\item Oferecer uma mudança de perspectiva das possibilidades
  composicionais derivadas dessa prática;
\item Montar um grupo de estudo que se reuna ao longo do semestre com
  o fim de promover um concerto com obras compostas como desdobramento
  do curso.
\end{itemize}

\section{Conteúdo programático}

O curso constará de 5 aulas cujo conteúdo programático será como
segue:

\begin{itemize}
\item Princípios de Acústica;
\item Noções básicas de informática em música;
\item Fundamentos da música eletroacústica;
\item Música concreta;
\item Música eletrônica;
\item Apreciação de obras paradigmáticas no contexto da música
  eletroacústica;
\item Introdução à síntese sonora;
\item Técnicas de gravação (melhor aproveitamento de acústica de salas e microfones);
\item Programas livres para síntese, interação, gravação e
  processamento de áudio digital: Audacity, Rosegarden, SND, Paul
  Extreme, CSound, Pure Data, entre outros;
\item Realizar o processamento de áudio através do Software Ardour a
  partir de suas próprias ferramentas e das oferecidas através de
  plug-ins LADSPA;
\item Mixagem e masterização;
\item Conceito de espacialização sonora;
\item Realizar exercícios práticos relacionados aos temas abordados.
\end{itemize}

\section{Metodologia}

Cada aula será dividida em dois blocos principais; no primeiro serão
apresentados aspectos teóricos e gerais do tópico em questão; no
segundo bloco serão corrigidos exercícios e dúvidas individuais.

As aulas serão ministradas no estúdio de música eletroacústica Genos e
utilizarão de seu equipamento em benefício do curso. Ao final de cada
aula serão propostos exercícios e tarefas para aprofundar o
conhecimento adquirido. As tarefas poderão ser feitas em casa, no caso
dos alunos que possuem computador, ou no próprio Genos, em horários
marcados na primeira aula.  Um fórum na internet (em
\url{http://wiki.genos.mus.br/cursoaudio}) foi criado onde os alunos
podem tirar dúvidas e pesquisar em links externos.

\section{Vagas}

A turma deverá ter entre 3 e 12 alunos, que deverão fazer inscrição
prévia.

\section{Das Aulas}

As aulas acontecerão no laboratório do GenosLab, na Escola de Música
da UFBa, diariamente durante uma semana. Como a carga horária diária
será de duas horas, o curso terá um total de dez horas aula.

\nocite{menezes96,dodge97:_comput_music,oliveira1995imp,menezes06:_music_maxim,vasconcelos02:_music_e_organ,laske91:_towar_epist_compos,stravinsky96:_music_em,Kramer1988,menezes01:_music_em_palav_e_sons,moore90:_elemen_of_comput_music,schaeffer93:_tratad_dos_objet_music,boulanger00:_csoun_book}

\bibliographystyle{achicago}
\bibliography{mestrado}

\end{document}



