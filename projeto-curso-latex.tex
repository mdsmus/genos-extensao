\documentclass[12pt,brazil]{article}
\usepackage[utf8x]{inputenc}
\usepackage{babel}
\usepackage[top=25mm,bottom=25mm,left=25mm,right=25mm]{geometry}
\usepackage{url}

\newcommand{\sigla}[1]{\textsc{#1}}
\newcommand{\eng}[1]{\textit{#1}}
\newcommand{\prog}[1]{\textsf{#1}}

\begin{document}
\selectlanguage{brazil}

\title{Escrevendo documentos acadêmicos com \LaTeX}
\author{Pedro Kröger}
\maketitle

\thispagestyle{empty}

\section{Introdução}

LaTeX é um sistema de tipografia de alta qualidade otimizado para
documentos estruturados e acadêmicos. Ele permite a criação fácil de
artigos, livros, relatórios, cartas, e apresentações. Esse curso vai
abordar os principais recursos para produzir documentos de maneira
profissional e produtiva, incluindo o gerenciamento de bibliografias
com BibTeX, uso de templates diversos como do SBC, ACM, e ANPPOM, uso
de figuras, referência cruzadas, código fonte, tabelas e fórmulas
matemáticas.

\section{Justificativa}

Apesar de existir a disciplina ``Informática em Música'' ela é
direcionada apenas a alunos da pós-graduação. Com esse curso
pretendemos 

\section{Objetivos}

Os objetivos do curso são:

\begin{itemize}
\item fornecer os conceitos e técnicas básicas do \LaTeX
\item apresentar elementos de texto como seções, fontes, ambientes
  para itemização, enumeração, citações, figuras e tabelas
\item apresentar como abstrair comandos com o \LaTeX
\item apresentar como usar estilos diferentes para um mesmo documento
\item apresentar como gerenciar referências cruzadas e bibliografias
  em documentos acadêmicos
\end{itemize}

\section{Conteúdo programático} 

O curso constará de 5 aulas cujo conteúdo programático será como
segue:

\begin{enumerate}
\item introdução básica ao Kile
\item criando um documento mínimo
\item introdução básica ao LaTeX: o que é e para que serve
\item comandos do LaTeX
\item caracteres especiais
\item títulos, capítulos e seções
\item comandos para fontes
\item ambientes para itemização, enumeração, descrição, código-fonte, e citações
\item inclusão de gráficos
\item tabelas, figuras
\item criando comandos
\item referências cruzadas
\item bibliografia com BibTeX
\item edição de fórmulas matemáticas
\item usando estilos LaTeX da ANPPOM e SBC
\item modificação do layout
\end{enumerate} 

\section{Metodologia}

Cada aula será dividida em dois blocos principais. No primeiro serão
apresentados aspectos teóricos e gerais do tópico em questão. No
segundo bloco os alunos praticarão os tópicos abordados. As aula serão
ministradas na sala 151 do Instituto de Matemática que possui
computadores suficientes para prática dos alunos. Durante as aulas
serão propostos exercícios e tarefas para aprofundar o conhecimento
adquirido.

\section{Referências bibliográficas para o curso}

A principal fonte bibliográfica será a apostila disponível na internet
``Introdução ao LATEX2''\footnote{Disponível em
  \url{www.ctan.org/tex-archive/info/lshort/portuguese/ptlshort.pdf}.}


\section{Clientela}
O curso se dirige aos alunos de música e ciência da computação, mas
alunos de outras áreas são bem vindos. Os alunos devem estar cursando
ou ter concluído o curso superior.
  
Os alunos deverão possuir algum conhecimento básico de como utilizar
um sistema operacional (Windows, MacOs, Linux) já que tais
conhecimentos \textbf{não} serão vistos durante o curso.

\section{Duração}

O curso terá 5 aulas de 2 horas.
 
\section{Vagas}

As turmas não devem ultrapassar a quantidade máxima de 25 e mínima de
10 alunos.

\section{Material necessário}
\label{sec:material}

Para o curso serão necessários os seguintes equipamentos:

Para a sala de aula:
\begin{enumerate}
\item 25 computadores (da sala 151 do Instituto de Matemática)
\item Datashow ou projetor similar
\item Quadro branco e canetas
\end{enumerate}

\end{document}
